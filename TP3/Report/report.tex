\documentclass[11pt,a4paper]{report}

\usepackage[utf8]{inputenc}
\usepackage[portuges]{babel}
\usepackage{indentfirst}
\usepackage{graphicx}
\usepackage{float}
\usepackage{caption}
\usepackage{subcaption}
\usepackage[T1]{fontenc}
\usepackage{listings}
\usepackage{amsmath}
\usepackage{mathtools}
\usepackage{tikz}
\renewcommand{\familydefault}{\sfdefault}

% packages que adicionei do stor
\usepackage{xspace}
\setlength{\oddsidemargin}{-1cm}
\setlength{\textwidth}{18cm}
\setlength{\headsep}{-1cm}
\setlength{\textheight}{23cm}

\title{Processamento de Linguagens (3º ano de Curso)\\
	\textbf{Trabalho Prático Nº3 (YACC)}\\ Relatório de Desenvolvimento}
\author{Diogo Braga\\ A82547 \and João Silva\\ A82005 \and Ricardo Caçador\\ A81064}
\date{\today}

\begin{document}

\maketitle

\begin{abstract}
	Neste relatório é apresentada a resolução de um exercício referente ao TP3, que tem como principais objetivos:
	\begin{itemize}
		\item aumentar a experiência de uso do ambiente \textbf{Linux}, da linguagem imperativa  \textbf{C}, e de algumas ferramentas de apoio à programação;
 		\item rever e aumentar a capacidade de escrever \textit{gramáticas independentes de contexto (GIC)}, que satisfaçam a condição LR(), para criar Linguagens de Domínio Específico (DSL);
 		\item desenvolver processadores de linguagens segundo o método da \textit{tradução dirigida pela sintaxe}, suportado numa \textit{gramática tradutora (GT)};
 		\item utilizar geradores de compiladores como o par \textbf{flex/yacc}.
	\end{itemize}
\end{abstract}

\tableofcontents

\newpage

\chapter{Introdução}
\label{chap:intro}

Seguindo a fórmula \emph{exercício = (N\_Alu\% 6)  +  1} e o número de aluno mais alto presente no nosso grupo (82547), o enunciado correspondente é o \textbf{6  -  Construtor de diaporama}.

///////////////////////// Descrever o problema aqui

//////////////////////

Com este relatório pretendemos apresentar as nossas opções e algoritmos desenvolvidos para a realização do gerador de diaporamas. Pretendemos também apoiar aquelas que foram as nossas soluções, com conhecimento obtido nas aulas teóricas.

Para uma melhor visão do que irá ser abordado neste relatório deixamos uma breve descrição daquilo que foi feito. No segundo capítulo foi feita uma análise informal e uma especificação dos requisitos deste projeto. No terceiro capítulo foi realizado o desenho da conceção no qual estão envolvidos os algoritmos e estruturas de dados usados. No quarto capítulo mostramos alguns exemplos de implementações e vários resultados de testes realizados. Por último no capítulo 5 fazemos uma retrospetiva do trabalho realizado e concluímos.


\chapter{Análise e Especificação}
\label{chap:analise}

//////////////////// FALTA FAZER

\section{Análise e Especificação dos Requisitos}

//////////////////// FALTA FAZER

\chapter{Conceção/desenho da Resolução}
\label{chap:concecao}

//////////////////// FALTA FAZER

\section{Algoritmos}

//////////////////// FALTA FAZER

\chapter{Codificação e Testes}
\label{chap:codificacao}

//////////////////// FALTA FAZER

\section{Estruturas de Dados}

//////////////////// FALTA FAZER

\section{Alternativas, Decisões e Problemas de Implementação}

//////////////////// FALTA FAZER

\section{Testes realizados e Resultados}

//////////////////// FALTA FAZER

\chapter{Extras}
\label{chap:extras}

//////////////////// FALTA FAZER

\chapter{Conclusão}
\label{chap:concl}

//////////////////// FALTA FAZER

\end{document}
