\documentclass[11pt,a4paper]{report}

\usepackage[utf8]{inputenc}
\usepackage[portuges]{babel}
\usepackage{indentfirst}
\usepackage{graphicx}
\usepackage{float}
\usepackage{caption}
\usepackage{subcaption}
\usepackage[T1]{fontenc}
\usepackage{listings}
\usepackage{amsmath}
\usepackage{mathtools}
\renewcommand{\familydefault}{\sfdefault}

% packages que adicionei do stor
\usepackage{xspace}
\setlength{\oddsidemargin}{-1cm}
\setlength{\textwidth}{18cm}
\setlength{\headsep}{-1cm}
\setlength{\textheight}{23cm}

\title{Processamento de Linguagens (3º ano de Curso)\\
	\textbf{Trabalho Prático Nº1}\\ Relatório de Desenvolvimento}
\author{Diogo Braga\\ A82547 \and João Silva\\ A82005 \and Ricardo Caçador\\ A81064}
\date{\today}

\begin{document}

\maketitle

\begin{abstract}
	Neste relatório é apresentada a resolução de um exercício referente ao TP1, que tem como objetivos a utilização de Expressões Regulares para descrição de padrões de frases, e a utilização do Flex para gerar filtros de texto em C.
Outro objetivo será ainda desenvolver, apartir de ERs, sistemática e automaticamente Processadores de Linguagens Regulares, que filtrem ou transformem textos com base no conceito de regras de produção Condição-Ação.
\end{abstract}

\tableofcontents

\newpage

\chapter{Introdução}
\label{chap:intro}

Seguindo a fórmula \emph{exe= (N\_Alu\% 7) + 1} e o número de aluno mais baixo presente no nosso grupo (81064), o enunciado correspondente é o \textbf{5 - WikiQuotes: provérbios}.

Este enunciado apresenta-nos um tema relacionado com várias citações de vários autores, vários provérbios de línguas diferentes, e ainda provérbios que são variações dos originais e alguns que são adulterações dos originais. Este tipo de material está armazenado num ficheiro \emph{XML}, onde se encontra muitas vezes organizado e outras vezes não tão bem organizado.

Ao longo deste trabalho produzimos essencialmente 3 filtros cada um com um objetivo bem delimitado. O primeiro filtro atua sobre as citações contidas em páginas de autor, o segundo filtro retém todas as citações cujo título da página comece com "Provérbios" e, por último, o terceiro filtro procura um padrão de provérbios que estejam adulterados. Neste último caso optamos também por filtrar os provérbios que são variações. De referir ainda que ligamos os provérbios adulterados e os que são variações ao respetivo provérbio original. O pergunta 4 do enunciado, o grupo optou por não a fazer separadamente dos outros filtros pelos que cada filtro apresenta estatísticas sobre o seu trabalho.

Com este relatório pretendemos apresentar as nossas opções, algoritmos desenvolvidos e ainda estruturas utilizadas para a realização de cada filtro. Pretendemos também apoiar aquelas que foram as nossas soluções, com conhecimento obtido nas aulas teóricas, relembrando por exemplo o caso dos autómatos.

Para uma melhor visão do que irá ser abordado neste relatório deixamos uma breve descrição daquilo que foi feito. No segundo capítulo foi feita uma análise informal e uma especificação dos requisitos deste projeto. No terceiro capítulo foi realizado o desenho da conceção no qual estão envolvidos os algoritmos e estruturas de dados usados. No quarto capítulo mostramos alguns exemplos de implementações e vários resultados de testes realizados. Por último no capítulo 5 fazemos uma retrospetiva do trabalho realizado e concluímos.



\chapter{Análise e Especificação}
\label{chap:analise}

Analisando o problema como um todo o que podemos encontrar aquando da observação do ficheiro \textbf{ptwikiquote-20190301-pages-articles.xml}, são várias meticulosidades no que toca à organização das páginas, títulos, citações, provérbios, etc. Contudo o nosso objetivo central é filtrar somente o que achamos necessário para respeitar os requisitos impostos pelo enunciado e descritos nas subsecções seguintes.

Nas secções seguintes serão apresentados os objetivos de cada alínea do exercício e ainda as observações que foram feitas ao ficheiro que contém todo o \emph{XML}, por forma a pensar que casos iríamos ter futuramente e começarmos a delinear uma arquitetura duma possível solução.

\section{Análise e Especificação dos Requisitos}
\subsection{Lista de Citações}

Na primeira alínea do exercício, era requerido um filtro que criasse uma lista de citações com o respetivo autor, somente se estas citações se encontrarem numa página de autor.

Ao proceder à análise do ficheiro \emph{XML} reparamos que uma página seria de autor se contivesse o seguinte cenário dentro da mesma:

\begin{figure}[H]
\centering
\includegraphics[scale=0.75]{pagina_autor.png}
\caption{Estrutura descritiva de um autor.}
\label{img:pagina_autor}
\end{figure}

Obviamente existem páginas que contém informações mais volumosas sobre um autor e outras páginas que contém menos. Posto isto reparamos que o campo \textbf{Nome}, pode ou não estar preenchido. Para os casos em que não está preenchido, o grupo achou por bem não considerar que a página se trataria de um autor, ignorando todos as citações na página contidas. O campo \textbf{Nome} possuía ainda outro pormenor. Muitas vezes surge referenciado noutra língua como por exemplo espanhol (\emph{Nombre}) e inglês (\emph{Name}).

Em páginas de autor investigámos de que forma aparecem a maior parte das citações que estes fazem. Chegamos à conclusão que na maior parte dos casos as citações surgem da seguinte forma:

\begin{figure}[H]
\centering
\includegraphics[scale=0.75]{quote.png}
\caption{Estrutura duma citação.}
\label{img:quote}
\end{figure}

Conseguimos identificar que a maior parte das citações de autores se encontram no meio de \textbf{\&quot;} e apenas teríamos de filtrar o que estivesse entre duas marcas deste tipo. Mas com a análise de mais casos deparámo-nos com situações desta natureza: 

\begin{figure}[H]
\centering
\includegraphics[scale=0.6]{marca_dentro_quote.png}
\caption{Exemplo de quatro marcas dentro duma citação.}
\label{img:marca_in_quote}
\end{figure}

Contudo surgiram muitos mais casos durante esta análise podemos referir casos em que citações não terminam com a marca \textbf{\&quot;} mas em vez disso acabam com por exemplo \textbf{::-}, ou \textbf{:-}, e ainda \textbf{**}.

Estes casos e muitos mais serão tratados nos capítulos seguintes.


\subsection{Lista de Provérbios}

Nesta segunda alínea do exercício, era requerido um filtro que criasse uma lista de provérbios (citações contidas em páginas cujo título começa por "Provérbios").


\subsection{Lista de Provérbios Adulterados}

Nesta terceira alínea do exercício, era requerida a listagem dos provérbios "adulterados" e o seu original. Era suposto procurar um padrão que identificasse estes provérbios.

\subsection{Estatísticas dos elementos encontrados}

Nesta quarta alínea do exercício, era requerida a apresentação duma estatísticas referentes ao que foi filtrado nas alíneas anteriores.



\chapter{Concepção/desenho da Resolução}
\label{chap:concepcao}

\section{Estruturas de Dados}
\subsection{Lista de Citações}

Quando começamos a desenvolver os primeiros filtros para testar com o ficheiro de input, deparamo-nos com a situação de existir mais que uma página para um mesmo autor. Seria portanto ideal que armazenássemos todas as citações referentes a um autor numa \textbf{Tabela de Hash} e íamos inserindo a esta as citações que aparecessem em diferentes páginas referentes a um mesmo autor.

Adotanto esta abordagem temos que filtrar tudo o que é necessário e só no fim despejar para o ecrã, de forma organizada, tudo aquilo que armazenamos na estrutura.

Desta forma criamos uma estrutura \textbf{TodasCitações}, que contém uma \textbf{GHashTable} cuja chave é o nome do autor e o valor é uma outra estrutura \textbf{Autor} que contém um nome e uma lista de citações. Estas estruturas são apresentadas nas seguintes imagens.

\begin{figure}[H]
\centering
\includegraphics[scale=0.9]{TodasCitacoes.png}
\caption{Estrutura TodasCitacoes.}
\label{img:todas_citacoes}
\end{figure}

\begin{figure}[H]
\centering
\includegraphics[scale=0.9]{Autor.png}
\caption{Estrutura Autor.}
\label{img:autor}
\end{figure}


\section{Algoritmos}
\subsection{Lista de Citações}


\subsection{Lista de Provérbios}


\subsection{Lista de Provérbios Adulterados}



\chapter{Codificação e Testes}
\label{chap:codificacao}

\section{Alternativas, Decisões e Problemas de Implementação}
\subsection{Lista de Citações}


\subsection{Lista de Provérbios}


\subsection{Lista de Provérbios Adulterados}


\section{Testes realizados e Resultados}
\subsection{Lista de Citações}


\subsection{Lista de Provérbios}


\subsection{Lista de Provérbios Adulterados}


\chapter{Conclusão}
\label{chap:concl}


\end{document}
